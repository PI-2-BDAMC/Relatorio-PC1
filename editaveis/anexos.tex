\begin{anexosenv}

\partanexos

\chapter{Documento de Visão}

\subsection{Introdução}

\subsubsection{Finalidade}

A finalidade deste documento é fornecer uma visão geral da aplicação de aquisição e manipulação dos dados do motor a combustão, apresentando uma visão das macro-funcionalidades do software. Além disso, objetiva-se apresentar as razões pelas quais o sistema será construído.

\subsubsection{Escopo}

A aplicação destina-se ao suporte ao usuário da bancada, com intuito de fornecer a visualização de informações a cerca do funcionamento do motor no momento da análise de forma gráfica. Tais informações são: Temperatura do óleo do motor; Temperatura do ar no coletor de admissão; Pressão do ar no coletor de admissão; Informações de emissão e mistura sonda/lâmbda.

\subsection{Posicionamento}

\subsubsection{Descrição do problema}

\begin{table}[h!]
	\centering
	\caption{Descrição do problema.}
	\label{descricaoproblema}
	\begin{tabular}{|l|l|}
		\hline
		\textbf{O problema de}                                                    & \begin{tabular}[c]{@{}l@{}}Dificuldade de se analisar os parâmetros e características de um,\\ motor em funcionamento.\end{tabular}                                      \\ \hline
		\textbf{Afeta}                                                            & Estudantes e professores do curso de engenharia automotiva.                                                                                                              \\ \hline
		\textbf{Cujo impacto é}                                                   & \begin{tabular}[c]{@{}l@{}}Impossibilidade, por parte dos alunos, de conhecer e analisar\\ parâmetros e características de um motor a combustão na prática.\end{tabular} \\ \hline
		\textbf{\begin{tabular}[c]{@{}l@{}}Uma boa solução \\ seria\end{tabular}} & \begin{tabular}[c]{@{}l@{}}Utilizar recursos gráficos, por meio de um software, para dar \\ suporte na análise dos dados de um motor em funcionamento.\end{tabular}      \\ \hline
	\end{tabular}
\end{table}

\subsubsection{Sentença de posição do produto}

\begin{table}[h!]
	\centering
	\caption{Sentença de posição do produto}
	\label{my-label}
	\begin{tabular}{|l|l|}
		\hline
		\textbf{Para}            & Compor a bancada de análise                                                                                                                                                              \\ \hline
		\textbf{Que}             & \begin{tabular}[c]{@{}l@{}}Necessita de um sistema de software para dar suporte na visualização\\ das informações do motor.\end{tabular}                                                 \\ \hline
		\textbf{O}               & Software de Aquisição e Processamento de Dados de Motor                                                                                                                                  \\ \hline
		\textbf{Que}             & Auxiliará o usuário da bancada a visualizar as informações do motor                                                                                                                      \\ \hline
		\textbf{Ao contrário de} & Realizar as análises com o veículo completo                                                                                                                                              \\ \hline
		\textbf{A aplicação}     & \begin{tabular}[c]{@{}l@{}}Promoverá uma interface gráfica entre o usuário da bancada e o motor \\ ao qual apresentará as informações referentes ao motor em funcionamento.\end{tabular} \\ \hline
	\end{tabular}
\end{table}


\subsubsection{Resumo dos envolvidos}


\subsection{Decrição dos Envolvidos e dos Usuários}


\chapter{Segundo Anexo}

Texto do segundo anexo.

\end{anexosenv}

